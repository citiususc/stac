\chapter{Diseño e implementación}
Una vez diseñada la arquitectura de la plataforma, es necesario detallar el proceso de creación de cada una de las unidades de las que consta la aplicación. Para ello, en primer lugar, se explicará como está organizado y desarrollado el módulo de Python de los test estadísticos. En segundo lugar, se detallará el diseño de la API REST desde el punto de vista de las URIs que conforman cada servicio web. Por último, se detallará la capa superior de la arquitectura, explicando cómo será la interfaz gráfica mediante prototipado y analizando su nivel de usabilidad mediante los principios heurísticos de Nielsen. Así mismo, hará uso de UML para especificar los diagramas de secuencia para cada historia de usuario.

\section{Módulo Python test estadísticos}
El módulo de Python de los test estadísticos o módulo STAC (Statistical Tests for Algorithm Comparison) está formado por los siguientes archivos:
\begin{itemize}
\item \textbf{tests\_parametricos.py:} contiene el test paramétrico ANOVA y el test POST-HOC de Bonferroni para el test de ANOVA.
\item \textbf{tests\_no\_parametricos.py:} contiene todos los test no paramétricos (Wilcoxon, Friedman, etc.) indicados en los objetivos del proyecto, excepto aquello que ya se encuentran en la librería SciPy, tal y como se indicó en la sección \ref{libreriastest}.
\item \textbf{\_\_init\_\_.py:} para que los archivos de test puedan ser importados todos a la vez, es posible empaquetarlos juntos. En este fichero se indica qué se importará al importar el módulo.
\end{itemize}
Este módulo, como se había comentado en la sección \ref{nonparametric} del capítulo \ref{arquitecturayh}, tiene dependencias con la librería SciPy (con el módulo \textit{stats}), así como con el paquete NumPy. La funcionalidad de SciPy sirve a este proyecto para poder hallar parámetros básicos que serán necesarios para obtener todos los datos necesarios en los test.

El diseño del los test estadísticos está determinado por los argumentos de entrada y valores de salida cada test.

\subsection{Test paramétricos}

\noindent
\textbf{Prototipo de la función del test ANOVA:}

\texttt{anova\_test(matriz\_datos, alpha=0.05)}

\begin{itemize}
\item \textbf{``matriz\_datos":} lista de listas de conjuntos de datos. Cada conjunto de datos representa los resultados obtenidos por los algoritmos al ser aplicados sobre un problema concreto.
\item \textbf{``alpha":} nivel de significación o error tipo I (probabilidad de rechazar la hipótesis nula siendo cierta).
\end{itemize}

\noindent
\textbf{Datos devueltos:}

Diccionario que contiene los siguientes datos:

\begin{itemize}
\item \textbf{``resultado":} \textit{true} ó \textit{false}, que indica que el contraste es estadísticamente significativo o estadísticamente no significativo, dependiendo de si el p\_valor es menor que el nivel de significación.
\item \textbf{``p\_valor":} probabilidad de obtener un valor al menos tan extremo como el estadístico hallado anteriormente suponiendo la hipótesis nula cierta.
\item \textbf{``estadístico":} valor del estadístico en cuestión.
\item \textbf{``variaciones":} lista con las variaciones o sumas de cuadrados total, entre tratamientos y variación del error (SCT, SCTR, SCE).
\item \textbf{``grados\_libertad":} lista con los grados de libertad totales, entre tratamientos y del error (GLT, GLTR, GLE).
\item \textbf{``cuadrados\_medios":} lista con los cuadrados medios (suma cuadrados / grados libertad) total, entre tratamientos y variación del error (SCT, SCTR, SCE).
\item \textbf{``medias\_algoritmos":} lista con las medias de los datos de cada tratamiento o algoritmo.
\item \textbf{``media\_general":} media de la lista de medias de los algoritmos.
\end{itemize}

\noindent
\textbf{Prototipo de la función test POST-HOC Bonferroni:}

\texttt{bonferroni\_test(nombres\_algoritmos, medias\_algoritmos, cuadrado\_medio\_error, N, alpha=0.05)}

\begin{itemize}
\item \textbf{``nombres\_algoritmos":} lista de los nombres de los algoritmos.
\item \textbf{``medias\_algoritmos":} lista con las medias de los datos de cada tratamiento o algoritmo.
\item \textbf{``cuadrado\_medio\_error":} valor del cuadrado medio del error (suma cuadrados error / grados libertad error).
\item \textbf{``N":} número de conjuntos de datos.
\item \textbf{``alpha":} nivel de significación o error tipo I (probabilidad de rechazar la hipótesis nula siendo cierta).
\end{itemize}

\noindent
\textbf{Datos devueltos:}

Diccionario que contiene los siguientes datos:

\begin{itemize}
\item \textbf{``resultado":} lista de valores True ó False, que indica si cada uno de los contrastes es o no estadísticamente significativo, dependiendo de si cada uno de los p\_valores es menor que el nivel de significación.
\item \textbf{``p\_valores":} lista de probabilidades de obtener un valor al menos tan extremo como cada uno de los estadísticos hallados suponiendo la hipótesis nula cierta.
\item \textbf{``valores\_t":} lista de valores o estadísticos para cada comparación.
\item \textbf{``p\_valores ajustados":} lista de los p\_valores de las comparaciones ajustados a toda la familia de comparaciones.
\item \textbf{``alpha":} nivel de significación (modificado según el valor de m).
\item \textbf{``comparaciones":} lista donde cada elemento es un texto que contiene los nombres de los dos algoritmos involucrados en la comparación tal como “algoritmoA vs algoritmoB”. Se ordena según del p\_valor de la comparación.
\end{itemize}

\subsection{Test no paramétricos}

\noindent
\textbf{Prototipos de las funciones test no paramétricos de ranking:}

\texttt{friedman\_test(nombres\_algoritmos, matriz\_datos, alpha=0.05, tipo=0)}

\texttt{iman\_davenport\_test(nombres\_algoritmos, matriz\_datos, alpha=0.05, tipo=0)}

\texttt{friedman\_rangos\_alineados\_test(nombres\_algoritmos, matriz\_datos, alpha=0.05, tipo=0)}

\texttt{quade\_test(nombres\_algoritmos, matriz\_datos, alpha=0.05, tipo=0)}

\begin{itemize}
\item \textbf{``nombres\_algoritmos":} lista que contiene los nombres de los algoritmos y que será empleada para devolver el ranking de nombres de los algoritmos.
\item \textbf{``matriz\_datos":} lista de listas de conjuntos de datos. Cada conjunto de datos representa los resultados obtenidos por los algoritmos al ser aplicados sobre un problema concreto.
\item \textbf{``alpha":} nivel de significación o error tipo I (probabilidad de rechazar la hipótesis nula siendo cierta).
\item \textbf{``tipo":} indica si lo que se pretende es minimizar (en cuyo caso se establece a 0) o maximizar (se establece a 1).
\end{itemize}

\noindent
\textbf{Datos devueltos:}

Diccionario que contiene los siguientes datos:

\begin{itemize}
\item \textbf{``resultado":} True ó False, que indica que el contraste es estadísticamente significativo o estadísticamente no significativo, dependiendo de si el p\_valor es menor que el nivel de significación.
\item \textbf{``p\_valor":} probabilidad de obtener un valor al menos tan extremo como el estadístico hallado anteriormente suponiendo la hipótesis nula cierta.
\item \textbf{``estadistico":} valor del estadístico en cuestión.
\item \textbf{``nombres":} lista de los nombres de los algoritmos ordenados según los valores numéricos de los rankings medios obtenidos por los distintos algoritmos.
\item \textbf{``ranking":} lista de los valores de los rankings medios ordenados de menor a mayor (cuanto menor es el valor mejor es el dato.)
\end{itemize}

\noindent
\textbf{Prototipos de las funciones tests POST-HOC:}

\texttt{bonferroni\_dunn\_test(K, nombres, valores\_z, p\_valores, metodo\_control, alpha=0.05)}

\texttt{holm\_test(K, nombres, valores\_z, p\_valores, metodo\_control, alpha=0.05)}

\texttt{hochberg\_test(K, nombres, valores\_z, p\_valores, metodo\_control, alpha=0.05)}

\texttt{li\_test(K, nombres, valores\_z, p\_valores, metodo\_control, alpha=0.05)}

\texttt{finner\_test(K, nombres, valores\_z, p\_valores, metodo\_control, alpha=0.05)}

\texttt{nemenyi\_multitest(m, comparaciones, valores\_z, p\_valores, alpha=0.05)}

\texttt{holm\_multitest(m, comparaciones, valores\_z, p\_valores, alpha=0.05)}

\texttt{hochberg\_multitest(m, comparaciones, valores\_z, p\_valores, alpha=0.05)}

\texttt{finner\_multitest(m, comparaciones, valores\_z, p\_valores, alpha=0.05)}

\texttt{shaffer\_multitest(m, comparaciones, valores\_z, p\_valores, alpha=0.05)}

\begin{itemize}
\item \textbf{``K":} número de algoritmos (incluyendo método de control).
\item \textbf{``nombres":} lista de nombres de los algoritmos (con los que el método de control se compara) ordenados según los p\_valores.
\item \textbf{``valores\_z":} estadísticos calculados durante el test. Siguen una normal (0, 1) y están ordenados según los p\_valores.
\item \textbf{``p\_valores":} p-valores calculados para comparar con los niveles de significancia ajustados.
\item \textbf{``metodo\_control":} método de control del test, por convención es el test de menor ranking.
\item \textbf{``alpha":} nivel de significancia (probabilidad de error tipo 1) del test de ranking principal.
\item \textbf{``m":} número de comparaciones.
\item \textbf{``comparaciones":} nombres de las hipótesis contrastadas. Por ejemplo “algoritmoA vs algoritmoB”.
\end{itemize}

\noindent
\textbf{Datos devueltos:}

Diccionario que contiene los siguientes datos:                                              

\begin{itemize}
\item \textbf{``valores\_z"}
\item \textbf{``p\_valores"}
\item \textbf{``alpha" \space ó ``alphas":} Nivel de significación. Bonferroni-Dunn y Li devuelven un alpha modificado. El resto de tests devuelven una lista de alphas cuyos elementos son diferentes para cada comparación.
\item \textbf{``resultado":} lista de valores True ó False, que indica si cada uno de los contrastes es o no estadísticamente significativo, dependiendo de si cada uno de los p\_valores es menor que el nivel de significación.
\item \textbf{``p\_valores ajustados":} lista de los p\_valores de las comparaciones ajustados a toda la familia de comparaciones.
\end{itemize}

Para los tests con método de control, el diccionario también incluye \textbf{``metodo\_control"} y \textbf{``nombres"}. Para los tests de comparación múltiple, éste también incluye: \textbf{``comparaciones"}.

\section{API Servicios REST}
Como se había comentado en la sección \ref{bottle} del capítulo \ref{arquitecturayh}, Bottle permite la implementación de servicios web accesibles mediante una o más URIs (\textit{Uniform Resource Identifier} o identificador de recursos uniforme). El decorador \textit{route()} asigna una URI a un trozo de código (el propio servicio web) en lo que se denomina ruta. El diseño de la API REST está determinado por las URIs mediante las cuales se puede acceder a los servicios.

La plataforma cuenta con una API REST en la que se distinguen dos tipos de servicios web, diferenciados por la utilidad que tienen:
\begin{itemize}
\item Servicios de subida / consulta de ficheros de datos.
\item Servicios de acceso / ejecución de los test.
\end{itemize}

Para los ejemplos de URIs en el diseño se supone que todos los servicios escuchan en /api/. Esto se pudo hacer modificando el archivo que utiliza el módulo WSGI para cargar la aplicación (API REST) en el servidor web Apache. En el manual técnico del presente documento (manual de despliegue) se detallará el contenido de este archivo.

\textbf{Servicios de subida y consulta de ficheros:} Podemos ver el diseño de este tipo de servicios en los cuadros \ref{cuadro1} y \ref{cuadro2}. El primer decorador permite la subida y el almacenamiento del contenido de un fichero en el servidor. El método usado es ``POST", ya que se envían datos al servidor (el fichero) y éste devuelve el resumen HASH del contenido del fichero o error en el caso de que se haya producido algún error en el procesamiento del archivo o éste no se encuentre en el servidor. El segundo decorador, permite consultar los datos de un fichero en concreto. Tiene un parámetro de entrada no opcional \textit{id\_fichero} (resumen HASH del contenido del fichero). En caso de no existir un fichero con esa clave, se devuelve error. El método empleado es ``GET", ya que únicamente se reciben datos del servidor.

\begin{table}[H]
	\centering
	\begin{tabular}{|l|}
		\hline
		\multicolumn{1}{|c|}{\textbf{Decorador}} \\ \hline
		\texttt{@route('/fichero', method="POST")} \\ \hline
		\texttt{@route('/fichero/<id\_fichero>', method="GET")} \\ \hline
	\end{tabular}
	\caption{Decoradores Bottle.}
	\label{cuadro1}
\end{table}

\begin{table}[H]
	\centering
	\begin{tabular}{|l|c|}
		\hline
		\multicolumn{1}{|c|}{\textbf{Ejemplo URI}} & {\textbf{Método}} \\ \hline
		\texttt{http://localhost/api/fichero} & \texttt{POST} \\ \hline
		\texttt{http://localhost/api/fichero/id\_fichero} & \texttt{GET} \\ \hline
	\end{tabular}
	\caption{Ejemplos de URIs y métodos empleados.}
	\label{cuadro2}
\end{table}

\textbf{Servicios de acceso y ejecución de test:} Estos servicios son los encargados de dar acceso a la ejecución de los test estadísticos y los que devuelven los datos proporcionados por éstos (en formato JSON) al navegador. La función AJAX de jQuery se encargará de mostrar estos datos en la interfaz de usuario. El método empleado es GET, ya que los servicios web de los test únicamente devuelven datos (resultado de su ejecución).

Los cuadros \ref{cuadro3} y \ref{cuadro4} muestran el diseño para los test de evaluación de las condiciones paramétricas, los test paramétricos y el test de Wilcoxon (no paramétrico):

\begin{table}[H]
	\centering
	\begin{tabular}{|l|}
		\hline
		\multicolumn{1}{|c|}{\textbf{Decorador}} \\ \hline
		\texttt{@route('/nombre/<id\_fichero>', method="GET")} \\ \hline
		\texttt{@route('/nombre/<id\_fichero>/<alpha:float>', method="GET") } \\ \hline
	\end{tabular}
	\caption{Decoradores Bottle.}
	\label{cuadro3}
\end{table}

\begin{table}[H]
	\centering
	\begin{tabular}{|l|c|}
		\hline
		\multicolumn{1}{|c|}{\textbf{Ejemplo URI}} & {\textbf{Método}} \\ \hline
		\texttt{http://localhost/api/ttest/id\_fichero} & \texttt{GET} \\ \hline
		\texttt{http://localhost/api/ttest/id\_fichero/0.05} & \texttt{GET} \\ \hline
	\end{tabular}
	\caption{Ejemplos de URIs y métodos empleados.}
	\label{cuadro4}
\end{table}

El valor \textit{nombre} indica el nombre del test referenciado (p. ej. /ttest/...) Los parámetros utilizados son los siguientes:
\begin{itemize}
\item \textbf{alpha:} nivel de significación. Parámetro opcional cuyo valor por defecto es 0.05 (probabilidad de error tipo 1 más común).
\item \textbf{id\_fichero:} resumen HASH del contenido del fichero que identifica al mismo inequívocamente. Se trata de un parámetro no opcional.
\end{itemize}

Los cuadros \ref{cuadro5} y \ref{cuadro6} muestran el diseño para los test no paramétricos de ranking (como el test de Friedman, el test de Iman-Davenport, etc.):

\begin{table}[H]
	\centering
	\begin{tabular}{|l|}
		\hline
		\multicolumn{1}{|c|}{\textbf{Decorador}} \\ \hline
		{\small \texttt{@route('/nombre/<id\_fichero>/<test\_comparacion>', method="GET")}} \\ \hline
		{\small \texttt{@route('/nombre/<id\_fichero>/<alpha:float>/<test\_comparacion>', method="GET")}} \\ \hline
		{\small \texttt{@route('/nombre/<id\_fichero>/<tipo:int>/<test\_comparacion>', method="GET")}} \\ \hline
		{\small \texttt{@route('/nombre/<id\_fichero>/<alpha:float>/<tipo:int>/<test\_comparacion>', method="GET")}} \\ \hline
		{\small \texttt{@route('/nombre/<id\_fichero>', method="GET")}} \\ \hline
		{\small \texttt{@route('/nombre/<id\_fichero>/<alpha:float>', method="GET")}} \\ \hline
		{\small \texttt{@route('/nombre/<id\_fichero>/<tipo:int>', method="GET")}} \\ \hline
		{\small \texttt{@route('/nombre/<id\_fichero>/<alpha:float>/<tipo:int>', method="GET")}} \\ \hline
	\end{tabular}
	\caption{Decoradores Bottle.}
	\label{cuadro5}
\end{table}

\begin{table}[H]
	\centering
	\begin{tabular}{|l|c|}
		\hline
		\multicolumn{1}{|c|}{\textbf{Ejemplo URI}} & {\textbf{Método}} \\ \hline
		\texttt{http://localhost/api/friedman/id\_fichero/li\_test} & \texttt{GET} \\ \hline
		\texttt{http://localhost/api/friedman/id\_fichero/0.05/li\_test} & \texttt{GET} \\ \hline
		\texttt{http://localhost/api/friedman/id\_fichero/1/li\_test} & \texttt{GET} \\ \hline
		\texttt{http://localhost/api/friedman/id\_fichero/0.05/1/li\_test} & \texttt{GET} \\ \hline
		\texttt{http://localhost/api/friedman/id\_fichero} & \texttt{GET} \\ \hline
		\texttt{http://localhost/api/friedman/id\_fichero/0.05} & \texttt{GET} \\ \hline
		\texttt{http://localhost/api/friedman/id\_fichero/1} & \texttt{GET} \\ \hline
		\texttt{http://localhost/api/friedman/id\_fichero/0.05/1} & \texttt{GET} \\ \hline
	\end{tabular}
	\caption{Ejemplos de URIs y métodos empleados.}
	\label{cuadro6}
\end{table}

El valor \textit{nombre} indica el nombre del test de ranking referenciado (p. ej. /friedman/...) Los parámetros utilizados son los siguientes:
\begin{itemize}
\item \textbf{alpha:} nivel de significación. Parámetro opcional cuyo valor por defecto es 0.05 (probabilidad de error tipo 1 más común).
\item \textbf{tipo:} indica la función objetivo de los algoritmos a contrastar: minimización (0) / maximización (1). Es un parámetro opcional cuyo valor por defecto es 0 (minimización).
\item \textbf{test\_comparacion:} indica el nombre del test POST-HOC a aplicar en caso de que el resultado del test de ranking determine que existen diferencias significativas. Se trata de un parámetro opcional cuyo valor por defecto es ``bonferroni\_dunn\_test".
\item \textbf{id\_fichero:} resumen HASH del contenido del fichero que identifica al mismo inequívocamente. Se trata de un parámetro no opcional.
\end{itemize}

\section{Aplicación web}
\subsection{Prototipo de la interfaz gráfica}
Para la implementación de la interfaz gráfica se construyeron inicialmente varios prototipos visuales de diferentes aspectos de la plataforma para luego poder proceder a su implementación. Para elaborar estos prototipos, se tuvieron en cuenta las historias de usuario del cliente vistas en la sección \ref{hu_cliente} del capítulo \ref{analisisreq}. En las figuras \ref{fig:prot_home}, \ref{fig:prot_help}, \ref{fig:prot_test} y \ref{fig:prot_results} podemos ver los prototipos de la página principal, la página donde se muestra la ayuda, la página de selección de parámetros de los test y la página de visualización de los resultados respectivamente. Por otro lado, en el manual de usuario al final del presente documento se muestra en las figuras ... el resultado final de la implementación.

\begin{figure}[H]
\centering
\includegraphics[width=10cm,height=6cm]{figuras/prototipo_home.jpg}
\caption{Prototipo de página principal.}
\label{fig:prot_home}
\end{figure}

\begin{figure}[H]
\centering
\includegraphics[width=10cm,height=6cm]{figuras/prototipo_help.jpg}
\caption{Prototipo página de ayuda.}
\label{fig:prot_help}
\end{figure}

\begin{figure}[H]
\centering
\includegraphics[width=10cm,height=6cm]{figuras/prototipo_test.jpg}
\caption{Prototipo página selección de parámetros/opciones.}
\label{fig:prot_test}
\end{figure}

\begin{figure}[H]
\centering
\includegraphics[width=10cm,height=6cm]{figuras/prototipo_results.jpg}
\caption{Prototipo página visualización de resultados.}
\label{fig:prot_results}
\end{figure}

\subsection{Heurísticas de Nielsen}
En la interacción persona-ordenador, se siguen varios pasos para crear sistemas que sean amigables para el usuario. En el paso de evaluación, se pueden realizar pruebas de expertos, en las cuales lo más común es utilizar las heurísticas creadas por Jakob Nielsen para evaluar el diseño de la interfaz de usuario. Los 10 principios de diseño basados en el usuario, que definió Jakob Nielsen en 1990, siguen siendo un referente importante para evaluar la usabilidad de un sitio web.

\noindent
\textbf{Visibilidad del estado del sistema}

\textit{El sistema debe siempre mantener a los usuarios informados del estado del sistema, con una realimentación apropiada y en un tiempo razonable.}

\noindent
\textbf{Lenguaje de los usuarios}

\textit{El sistema debe hablar el lenguaje de los usuarios, con las palabras, las frases y los conceptos familiares, en lugar de que los términos estén orientados al sistema. Utilizar convenciones del mundo real, haciendo que la información aparezca en un orden natural y lógico.}

\noindent
\textbf{Control y libertad para el usuario}

\textit{Los usuarios eligen a veces funciones del sistema por error y necesitan a menudo una salida de emergencia claramente marcada, esto es, salir del estado indeseado sin tener que pasar por un diálogo extendido. Es importante disponer de deshacer y rehacer.}

\noindent
\textbf{Consistencia y estándares}

\textit{Los usuarios no deben tener que preguntarse si las diversas palabras, situaciones, o acciones significan la misma cosa. En general siga las normas y convenciones de la plataforma sobre la que se está implementando el sistema.}

\noindent
\textbf{Ayuda a los usuarios para reconocimiento, diagnóstico y recuperación de errores}

\textit{Los mensajes de error se deben expresar en un lenguaje claro (no haya códigos extraños), se debe indicar exactamente el problema, y deben ser constructivos.}

\noindent
\textbf{Prevención de errores}

\textit{Es importante prevenir la aparición de errores, mejor que generar buenos mensajes de error.}

\noindent
\textbf{Reconocimiento antes que cancelación}

\textit{El usuario no debería tener que recordar la información de una parte de diálogo para otra. Es mejor mantener objetos, acciones, y las opciones visibles que memorizar.}

\noindent
\textbf{Flexibilidad y eficiencia de uso}

\textit{Las instrucciones para el uso del sistema deben ser visibles o fácilmente accesibles siempre que se necesiten. Los aceleradores no vistos por el usuario principiante, mejoran la interacción para el usuario experto de tal manera que el sistema puede servir para usuarios inexpertos y experimentados. Es importante que el sistema permita personalizar acciones frecuentes.}

\noindent
\textbf{Estética de diálogos y diseño minimalista}

\textit{No deben contener información que sea inaplicable o se necesite raramente. Cada unidad adicional de información en un diálogo compite con las unidades relevantes de información y disminuye su visibilidad relativa.}

\noindent
\textbf{Ayuda general y documentación}

\textit{Aunque es mejor si el sistema se puede usar sin documentación, puede ser necesario disponer de ayuda y documentación. Esta ha de ser fácil de buscar, centrada en las tareas del usuario, tener información de las etapas a realizar y que no sea muy extensa.}

\subsection{Diagramas de secuencia}

\begin{figure}[H]
\centering
\includegraphics[width=11cm,height=8cm]{figuras/sec_subir_fichero.jpg}
\caption{Diagrama de secuencia de la subida de ficheros.}
\label{fig:sec_subir_fichero}
\end{figure}

\begin{figure}[H]
\centering
\includegraphics[width=11cm,height=8cm]{figuras/sec_consultar_fichero.jpg}
\caption{Diagrama de secuencia de la consulta de ficheros.}
\label{fig:sec_consultar_fichero}
\end{figure}
