%***************************************************************************************************************************

\chapter{Análisis de requisitos}
El análisis de requisitos es el primer paso técnico del proceso de ingeniería del software. Es aquí donde se refina la declaración general del ámbito del software en una especificación concreta que se convierte en la base de todas las actividades de ingeniería del software que siguen.

En este proyecto, hemos utilizado el enfoque de metodología de desarrollo ágil Scrum \cite{libroscrum}. Los detalles de esta metodología y la razón de su utilización en este proyecto se detallan en la sección \ref{metodologia} del capítulo 4. Para realizar el análisis de requisitos, a diferencia del enfoque de tradicional en el que los requisitos son descritos de una forma muy estricta y formal, esta metodología permite describir las necesidades del usuario de una manera más simple. Para ello, se establecen los requisitos mediante las \textbf{historias de usuario}. Las historias de usuario son requisitos escritos en un lenguaje coloquial bien directamente por el mismo cliente, o bien como un recordatorio posterior de las conversaciones mantenidas con el cliente. Consisten en una o dos frases en donde de una forma no precisa se detalla lo que el usuario requiere de la aplicación. Además, deben ser:

\begin{itemize}
\item \textbf{Independientes:} No depender de otras para su compleción.
\item \textbf{Negociables:} No son del todo claras y, por tanto se necesita discutir con los usuarios. Se concretan en los criterios de aceptación.
\item \textbf{Valoradas por el cliente:} Esto permite conocer en qué está más interesado el cliente y qué es más importante para la aplicación.
\item \textbf{Estimables:} Se puede establecer una valoración del tiempo que llevará completarlas.
\item \textbf{Pequeñas:} Para poder hacer una mejor estimación. Normalmente más de 2 días y menos de 1 semana.
\item \textbf{Verificables:} Se necesitan poder probar para saber si se ha completado con éxito. 
\end{itemize}

El formato a seguir es el siguiente:

\begin{center}
Como $<$tipo de usuario$>$, me gustaría $<$objetivo$>$, ya que $<$razón$>$
\end{center}

Además, a las historias de usuario se añaden criterios de aceptación y una prioridad.

Entre los beneficios de usar historias de usuario para elaborar los requisitos destaca que no se requiere elaborar una gran cantidad de documentos formales y por lo tanto se requiere menos tiempo para su administración. Por ello, permiten responder rápidamente a los requisitos cambiantes, algo a tener en cuenta sabiendo que normalmente los clientes o los usuarios finales con frecuencia no saben lo que necesitan desde un principio y es algo que se debe ir refinando a lo largo del proyecto.

Un nivel de abstracción mayor a las historias de usuario es dado por los denominados Epics. Los Epics son historias de usuario mucho más generales (a más alto nivel) que nos dan una primera idea del trabajo que puede estar involucrado en su definición. Los Epics se pueden desglosar en varias historias de usuario y están compuestos por un título o requerimiento muy general.

En este proyecto, primeramente se detallan los Epics y luego se describen las historias de usuario en las que se pueden desglosar. Para la prioridad de las historias de usuario, se han definido tres niveles en función del interés del cliente en las mismas:

\begin{itemize}
\item \textbf{Alta:} Indica que el cliente está muy interesado en el requerimiento en cuestión y que lo considera clave para la aplicación.
\item \textbf{Media:} Indica que el cliente está interesado en el requerimiento, pero no es tan importante para la aplicación. 
\item \textbf{Baja:} Indica aquellos requerimientos que, siendo favorables, puede prescindir de ellos.
\end{itemize}

%***************************************************************************************************************************

\section{Epics} \label{epics}
Los Epics son identificados mediante EP-x, donde x representa un número natural único.

\begin{itemize}
\item \textbf{EP-1:} Proporcionar una API REST de los tests estadísticos.
\item \textbf{EP-2:} Permitir realizar tests estadísticos paramétricos.
\item \textbf{EP-3:} Permitir realizar tests estadísticos no paramétricos.
\item \textbf{EP-4:} Permitir realizar tests estadísticos no paramétricos para evaluar las condiciones paramétricas.
\item \textbf{EP-5:} Permitir gestionar un fichero.
\item \textbf{EP-6:} Visualizar resultados tests.
\item \textbf{EP-7:} Permitir modificar opciones de los tests.
\item \textbf{EP-8:} Proporcionar una interfaz usable.
\end{itemize}

%***************************************************************************************************************************

\section{Historias de usuario}
Las historias de usuario se identifican mediante HU-x, donde x representa un número natural único. Cabe destacar que en este proyecto se consideran dos tipos de usuario: desarrollador y cliente. El primero, se considera debido a que una de las partes del proyecto consiste en desarrollar una API REST que hace disponible los tests vía web. Esta API la puede utilizar un desarrollador para realizar la interfaz web, con lo que es necesario que un desarrollador diga las necesidades que requiere de la API. El cliente, por otra parte, se refiere al usuario final de la plataforma, es decir, el analista de datos que quiere validar algoritmos de aprendizaje automático. Ambos tienen intereses diferentes y por tanto se pueden considerar por separado.

% ----------------------------------------------------- %

\subsection{Historias de usuario desarrollador}

% *****************HISTORIA USUARIO 1****************** %

Desglosando el Epic \textbf{EP-1:} Proporcionar una API REST de los tests estadísticos, se obtienen las siguientes historias de usuario:

\begin{table}[H]
	\begin{tabular}{| p{3cm}| p{11cm} |}
		\hline
		\multicolumn{2}{|c|}{\textbf{HU-1} - Acceder a tests} \\ \hline
		\textbf{Como:} & Desarrollador \\ \hline
		\textbf{Me gustaría:} & acceder a los tests como recursos independientes con parámetros opcionales \\ \hline
		\textbf{Ya que:} & esto facilita su utilización y los hace más flexibles \\ \hline
		\multirow{3}{11cm}{\textbf{C. Aceptación:}} & - Los tests estadísticos son accesibles individualmente como un recurso \\
		& - Los métodos de POST-HOC pueden ser también accedidos individualmente como un sub recurso dentro del recurso de acceso del test principal \\
		& - Se proporcionan varias URIs (identificador de recursos uniforme) para cada test cuyos parámetros (como el nivel de significación, la función objetivo o el test POST-HOC) son opcionales, teniendo un valor común por defecto \\ \hline
		\textbf{\textbf{Prioridad:}} & alta \\ \hline
	\end{tabular}
\end{table}

% *****************HISTORIA USUARIO 2****************** %

\begin{table}[H]
	\begin{tabular}{| p{3cm}| p{11cm} |}
		\hline
		\multicolumn{2}{|c|}{\textbf{HU-2} - Gestionar ficheros} \\ \hline
		\textbf{Como:} & Desarrollador \\ \hline
		\textbf{Me gustaría:} & poder gestionar un fichero como un recurso independiente \\ \hline
		\textbf{Ya que:} & esto facilita la subida y consulta de ficheros \\ \hline
		\multirow{2}{11cm}{\textbf{C. Aceptación:}} & - Los ficheros son accesibles individualmente como un recurso \\
		& - Los recursos de fichero son creados de forma individual \\ \hline 
		\textbf{\textbf{Prioridad:}} & alta \\ \hline
	\end{tabular}
\end{table}
	
% *****************HISTORIA USUARIO 3****************** %

\begin{table}[H]
	\begin{tabular}{| p{3cm}| p{11cm} |}
		\hline
		\multicolumn{2}{|c|}{\textbf{HU-3} - Devolver datos JSON} \\ \hline
		\textbf{Como:} & Desarrollador \\ \hline
		\textbf{Me gustaría:} & que los datos devueltos por los servicios de la API REST fuesen en formato JSON (JavaScript Object Notation) \\ \hline
		\textbf{Ya que:} & es un formato ligero para el intercambio de datos y además es muy simple, con lo que el análisis sintáctico sería más sencillo \\ \hline
		\textbf{C. Aceptación:} & - Los servicios de la API REST devuelven datos en formato JSON \\ \hline
		\textbf{\textbf{Prioridad:}} & media \\ \hline
	\end{tabular}
\end{table}

% *****************HISTORIA USUARIO 4****************** %

\begin{table}[H]
	\begin{tabular}{| p{3cm}| p{11cm} |}
		\hline
		\multicolumn{2}{|c|}{\textbf{HU-4} - Analizar datos subidos} \\ \hline
		\textbf{Como:} & Desarrollador \\ \hline
		\textbf{Me gustaría:} & que el fichero de datos subido al servidor tuviese siempre el formato csv \\ \hline
		\textbf{Ya que:} & es un formato abierto y sencillo para representar datos en forma de tabla y los tests reciben siempre los datos en forma de matriz, donde las filas representan los conjuntos de datos de cada problema y las columnas los resultados obtenidos de cada algoritmo \\ \hline
		\multirow{2}{11cm}{\textbf{C. Aceptación:}} & - El servicio de subida comprueba que los datos subidos siguen el estándar csv y están dispuestos de acuerdo a una convención establecida para el proyecto \\
		& - En caso de no cumplir el estándar se devuelve un error \\ \hline 
		\textbf{\textbf{Prioridad:}} & alta \\ \hline
	\end{tabular}
\end{table}
	
	
% *****************HISTORIA USUARIO 5****************** %

\begin{table}[H]
	\begin{tabular}{| p{3cm}| p{11cm} |}
		\hline
		\multicolumn{2}{|c|}{\textbf{HU-5} - Limitar ficheros subidos} \\ \hline
		\textbf{Como:} & Desarrollador \\ \hline
		\textbf{Me gustaría:} & establecer un límite de ficheros subidos \\ \hline
		\textbf{Ya que:} & así se evita tener en memoria ficheros muy antiguos que ya no se usan \\ \hline
		\textbf{C. Aceptación:} & - Los ficheros se almacenan en un diccionario con límite de elementos, de forma que cuando se llega al límite, el elemento con más tiempo en el diccionario se elimina \\ \hline
		\textbf{\textbf{Prioridad:}} & baja \\ \hline
	\end{tabular}
\end{table}

% *****************HISTORIA USUARIO 6****************** %

\begin{table}[H]
	\begin{tabular}{| p{3cm}| p{11cm} |}
		\hline
		\multicolumn{2}{|c|}{\textbf{HU-6} - Visualizar información tests} \\ \hline
		\textbf{Como:} & Desarrollador \\ \hline
		\textbf{Me gustaría:} & obtener información acerca de los tests estadísticos \\ \hline
		\textbf{Ya que:} & esto permite conocer qué hace cada test y evita confusiones a la hora de utilizar el servicio \\ \hline
		\multirow{2}{11cm}{\textbf{C. Aceptación:}} & - Los servicios de la API REST tienen información relativa al test (uso e hipótesis) \\
		& - El módulo de Python donde están implementados los tests tiene documentación en base a un generador de documentación relativa al test (uso, hipótesis, argumentos de entrada / salida, ...) \\ \hline 
		\textbf{\textbf{Prioridad:}} & media \\ \hline
	\end{tabular}
\end{table}

% ----------------------------------------------------- %

\subsection{Historias de usuario cliente}

% *****************HISTORIA USUARIO 7****************** %

Desglosando el Epic \textbf{EP-2:} Permitir realizar tests estadísticos paramétricos, se obtienen las siguientes historias de usuario:

\begin{table}[H]
	\begin{tabular}{| p{3cm}| p{11cm} |}
		\hline
		\multicolumn{2}{|c|}{\textbf{HU-7} - Realizar test de ANOVA} \\ \hline
		\textbf{Como:} & Cliente \\ \hline
		\textbf{Me gustaría:} & poder aplicar el test de ANOVA a mis datos \\ \hline
		\textbf{Ya que:} & es una prueba muy utilizada para determinar si la media de más de dos muestras son similares \\ \hline
		\multirow{3}{11cm}{\textbf{C. Aceptación:}} & - El módulo de Python tiene implementado el test ANOVA \\
		& - La plataforma muestra una sección para tests paramétricos donde se puede aplicar la prueba \\
		& - El test devuelve los datos: estadístico, el $p-valor$ y el resultado. En caso de ser estadísticamente significativo, también aparecerán los resultados del test POST-HOC de Bonferroni ($p-valores$, $p-valores$ ajustados, etc.) \\ \hline
		\textbf{\textbf{Prioridad:}} & alta \\ \hline
	\end{tabular}
\end{table}

% *****************HISTORIA USUARIO 8****************** %

\begin{table}[H]
	\begin{tabular}{| p{3cm}| p{11cm} |}
		\hline
		\multicolumn{2}{|c|}{\textbf{HU-8} - Realizar test T-test} \\ \hline
		\textbf{Como:} & Cliente \\ \hline
		\textbf{Me gustaría:} & poder aplicar el test T-test a mis datos \\ \hline
		\textbf{Ya que:} & aunque contrasta la misma hipótesis que ANOVA éste tiene otro estadístico y funciona únicamente para dos algoritmos, con lo que puede resultar interesante tener otro punto de vista \\ \hline
		\multirow{4}{11cm}{\textbf{C. Aceptación:}} & - El módulo de Python tiene implementado el test T-test \\
		& - La plataforma muestra una sección para tests paramétricos donde se puede aplicar la prueba \\
		& - El test devuelve los datos: estadístico, el $p-valor$ y el resultado \\
		& - Devuelve error en caso de que los datos tengan resultados de más de 2 algoritmos \\ \hline
		\textbf{\textbf{Prioridad:}} & media \\ \hline
	\end{tabular}
\end{table}

% *****************HISTORIA USUARIO 9****************** %

\clearpage
Desglosando el Epic \textbf{EP-3:} Permitir realizar tests estadísticos no paramétricos, se obtienen las siguientes historias de usuario:

\begin{table}[H]
	\begin{tabular}{| p{3cm}| p{11cm} |}
		\hline
		\multicolumn{2}{|c|}{\textbf{HU-9} - Realizar test de Wilcoxon} \\ \hline
		\textbf{Como:} & Cliente \\ \hline
		\textbf{Me gustaría:} & poder aplicar el test de Wilcoxon a mis datos \\ \hline
		\textbf{Ya que:} & es útil como alternativa al test T-test cuando los datos no cumplen con el supuesto de normalidad \\ \hline
		\multirow{4}{11cm}{\textbf{C. Aceptación:}} & - El módulo de Python tiene implementado el test de Wilcoxon \\
		& - La plataforma muestra una sección para tests no paramétricos donde se puede aplicar la prueba \\
		& - El test devuelve el estadístico, el $p-valor$,... incluyendo también la suma de los rangos positivos y la suma de los rangos negativos característicos del test de Wilcoxon \\
		& - Devuelve error en caso de que los datos tengan resultados de más de 2 algoritmos \\ \hline
		\textbf{\textbf{Prioridad:}} & alta \\ \hline
	\end{tabular}
\end{table}

% *****************HISTORIA USUARIO 10***************** %

\begin{table}[H]
	\begin{tabular}{| p{3cm}| p{11cm} |}
		\hline
		\multicolumn{2}{|c|}{\textbf{HU-10} - Realizar test de Friedman} \\ \hline
		\textbf{Como:} & Cliente \\ \hline
		\textbf{Me gustaría:} & poder aplicar el test de Friedman a mis datos \\ \hline
		\textbf{Ya que:} & es una prueba muy útil a la hora de comparar algoritmos (ya que trabaja asignando rankings) y determinar si existen diferencias entre ellos \\ \hline
		\multirow{3}{11cm}{\textbf{C. Aceptación:}} & - El módulo de Python tiene implementado el test de Friedman \\
		& - La plataforma muestra una sección para tests no paramétricos donde se puede aplicar la prueba \\
		& - El test devuelve los datos: estadístico, el $p-valor$, el resultado y el ranking \\ \hline
		\textbf{\textbf{Prioridad:}} & alta \\ \hline
	\end{tabular}
\end{table}

% *****************HISTORIA USUARIO 11***************** %

\begin{table}[H]
	\begin{tabular}{| p{3cm}| p{11cm} |}
		\hline
		\multicolumn{2}{|c|}{\textbf{HU-11} - Realizar test de Iman-Davenport} \\ \hline
		\textbf{Como:} & Cliente \\ \hline
		\textbf{Me gustaría:} & poder aplicar el test de Iman-Davenport a mis datos \\ \hline
		\textbf{Ya que:} & es una prueba cuyo estadístico es más ajustado que el de Friedman y por lo tanto es más potente y puede servir mejor para mis datos \\ \hline
		\multirow{3}{11cm}{\textbf{C. Aceptación:}} & - El módulo de Python tiene implementado el test de Iman-Davenport \\
		& - La plataforma muestra una sección para tests no paramétricos donde se puede aplicar la prueba \\
		& - El test devuelve los datos: estadístico, el $p-valor$, el resultado y el ranking \\ \hline
		\textbf{\textbf{Prioridad:}} & alta \\ \hline
	\end{tabular}
\end{table}

% *****************HISTORIA USUARIO 12***************** %

\begin{table}[H]
	\begin{tabular}{| p{3cm}| p{11cm} |}
		\hline
		\multicolumn{2}{|c|}{\textbf{HU-12} - Realizar test de los Rangos Alineados de Friedman} \\ \hline
		\textbf{Como:} & Cliente \\ \hline
		\textbf{Me gustaría:} & poder aplicar el test de Quade \\ \hline
		\textbf{Ya que:} & es una prueba que a diferencia de la de Friedman establece los rankings teniendo en cuentas las posibles interrelaciones que pueden haber entre los distintos conjuntos de datos o problemas, con lo que puede ser muy útil en ese sentido \\ \hline
		\multirow{3}{11cm}{\textbf{C. Aceptación:}} & - El módulo de Python tiene implementado el test de Rangos Alineados de Friedman \\
		& - La plataforma muestra una sección para tests no paramétricos donde se puede aplicar la prueba \\
		& - El test devuelve los datos: estadístico, el $p-valor$, el resultado y el ranking \\ \hline
		\textbf{\textbf{Prioridad:}} & alta \\ \hline
	\end{tabular}
\end{table}

% *****************HISTORIA USUARIO 13***************** %

\begin{table}[H]
	\begin{tabular}{| p{3cm}| p{11cm} |}
		\hline
		\multicolumn{2}{|c|}{\textbf{HU-13} - Realizar test de Quade} \\ \hline
		\textbf{Como:} & Cliente \\ \hline
		\textbf{Me gustaría:} & poder aplicar el test de Quade \\ \hline
		\textbf{Ya que:} & es una prueba que a diferencia de la de Friedman considera que algunos problemas son más difíciles, o que los resultados que obtienen los algoritmos sobre ellos son más distantes, con lo cual es de las pruebas de ranking más potentes y mejores que puedo aplicar sobre mis resultados \\ \hline
		\multirow{3}{11cm}{\textbf{C. Aceptación:}} & - El módulo de Python tiene implementado el test de Quade \\
		& - La plataforma muestra una sección para tests no paramétricos donde se puede aplicar la prueba \\
		& - El test devuelve los datos: estadístico, el $p-valor$, el resultado y el ranking \\ \hline
		\textbf{\textbf{Prioridad:}} & alta \\ \hline
	\end{tabular}
\end{table}

% *****************HISTORIA USUARIO 14***************** %

\begin{table}[H]
	\begin{tabular}{| p{3cm}| p{11cm} |}
		\hline
		\multicolumn{2}{|c|}{\textbf{HU-14} - Realizar test de Bonferroni-Dunn} \\ \hline
		\textbf{Como:} & Cliente \\ \hline
		\textbf{Me gustaría:} & poder aplicar el test de Bonferroni-Dunn \\ \hline
		\textbf{Ya que:} & es una prueba muy común que se realiza en caso de que los tests de ranking obtengan diferencias significativas para detectar diferencias concretas entre pares de algoritmos \\ \hline
		\multirow{3}{11cm}{\textbf{C. Aceptación:}} & - El módulo de Python tiene implementado el test de Bonferroni-Dunn con método de control y el test para comparaciones múltiples \\
		& - La plataforma muestra una sección para tests no paramétricos donde se puede aplicar la prueba tanto simple como comparación múltiple \\
		& - El test devuelve los resultados propios del test, como los $p-valores$ ajustados, los estadísticos, el método de control, etc. en caso de que el test de ranking principal sea estadísticamente significativo. Si se trata del POST-HOC simple, se realizarán $K-1$ comparaciones. En el caso de la prueba multitest, se realizarán $\frac{K(K-1)}{2}$ comparaciones \\ \hline
		\textbf{\textbf{Prioridad:}} & alta \\ \hline
	\end{tabular}
\end{table}

% *****************HISTORIA USUARIO 15***************** %

\begin{table}[H]
	\begin{tabular}{| p{3cm}| p{11cm} |}
		\hline
		\multicolumn{2}{|c|}{\textbf{HU-15} - Realizar test de Holm} \\ \hline
		\textbf{Como:} & Cliente \\ \hline
		\textbf{Me gustaría:} & poder aplicar el test de Holm \\ \hline
		\textbf{Ya que:} & se trata de una prueba de comparación a posteriori con más potencia que la de Bonferroni-Dunn y puede resultar más útil al rechazar posiblemente más hipótesis \\ \hline
		\multirow{3}{11cm}{\textbf{C. Aceptación:}} & - El módulo de Python tiene implementado el test de Holm con método de control y el test para comparaciones múltiples \\
		& - La plataforma muestra una sección para tests no paramétricos donde se puede aplicar la prueba tanto simple como comparación múltiple \\
		& - El test devuelve los resultados propios del test, como los $p-valores$ ajustados, los estadísticos, el método de control, etc. en caso de que el test de ranking principal sea estadísticamente significativo. Si se trata del POST-HOC simple, se realizarán $K-1$ comparaciones. En el caso de la prueba multitest, se realizarán $\frac{K(K-1)}{2}$ comparaciones \\ \hline
		\textbf{\textbf{Prioridad:}} & alta \\ \hline
	\end{tabular}
\end{table}

% *****************HISTORIA USUARIO 16***************** %

\begin{table}[H]
	\begin{tabular}{| p{3cm}| p{11cm} |}
		\hline
		\multicolumn{2}{|c|}{\textbf{HU-16} - Realizar test de Finner} \\ \hline
		\textbf{Como:} & Cliente \\ \hline
		\textbf{Me gustaría:} & poder aplicar el test de Finner \\ \hline
		\textbf{Ya que:} & se trata de una prueba de comparación a posteriori con más potencia que la de Holm y puede resultar más útil al rechazar posiblemente más hipótesis \\ \hline
		\multirow{3}{11cm}{\textbf{C. Aceptación:}} & - El módulo de Python tiene implementado el test de Finner con método de control y el test para comparaciones múltiples \\
		& - La plataforma muestra una sección para tests no paramétricos donde se puede aplicar la prueba tanto simple como comparación múltiple \\
		& - El test devuelve los resultados propios del test, como los $p-valores$ ajustados, los estadísticos, el método de control, etc. en caso de que el test de ranking principal sea estadísticamente significativo. Si se trata del POST-HOC simple, se realizarán $K-1$ comparaciones. En el caso de la prueba multitest, se realizarán $\frac{K(K-1)}{2}$ comparaciones \\ \hline
		\textbf{\textbf{Prioridad:}} & alta \\ \hline
	\end{tabular}
\end{table}

% *****************HISTORIA USUARIO 17***************** %

\begin{table}[H]
	\begin{tabular}{| p{3cm}| p{11cm} |}
		\hline
		\multicolumn{2}{|c|}{\textbf{HU-17} - Realizar test de Hochberg} \\ \hline
		\textbf{Como:} & Cliente \\ \hline
		\textbf{Me gustaría:} & poder aplicar el test de Hochberg \\ \hline
		\textbf{Ya que:} & se trata de una prueba de comparación a posteriori con más potencia que la de Finner y puede resultar más útil al rechazar posiblemente más hipótesis \\ \hline
		\multirow{3}{11cm}{\textbf{C. Aceptación:}} & - El módulo de Python tiene implementado el test de Hochberg con método de control y el test para comparaciones múltiples \\
		& - La plataforma muestra una sección para tests no paramétricos donde se puede aplicar la prueba tanto simple como comparación múltiple \\
		& - El test devuelve los resultados propios del test, como los $p-valores$ ajustados, los estadísticos, el método de control, etc. en caso de que el test de ranking principal sea estadísticamente significativo. Si se trata del POST-HOC simple, se realizarán $K-1$ comparaciones. En el caso de la prueba multitest, se realizarán $\frac{K(K-1)}{2}$ comparaciones \\ \hline
		\textbf{\textbf{Prioridad:}} & alta \\ \hline
	\end{tabular}
\end{table}

% *****************HISTORIA USUARIO 18***************** %

\begin{table}[H]
	\begin{tabular}{| p{3cm}| p{11cm} |}
		\hline
		\multicolumn{2}{|c|}{\textbf{HU-18} - Realizar test de Li} \\ \hline
		\textbf{Como:} & Cliente \\ \hline
		\textbf{Me gustaría:} & poder aplicar el test de Li \\ \hline
		\textbf{Ya que:} & se trata de una prueba de comparación a posteriori con más potencia que la de Hochberg y puede resultar más útil al rechazar posiblemente más hipótesis \\ \hline
		\multirow{3}{11cm}{\textbf{C. Aceptación:}} & - El módulo de Python tiene implementado el test de Li simple con método de control \\
		& - La plataforma muestra una sección para tests no paramétricos donde se puede aplicar la prueba \\
		& - El test devuelve los resultados propios del test, como los $K-1$ $p-valores$ ajustados, estadísticos, resultados, etc. \\ \hline
		\textbf{\textbf{Prioridad:}} & alta \\ \hline
	\end{tabular}
\end{table}

% *****************HISTORIA USUARIO 19***************** %

\begin{table}[H]
	\begin{tabular}{| p{3cm}| p{11cm} |}
		\hline
		\multicolumn{2}{|c|}{\textbf{HU-19} - Realizar test de Shaffer} \\ \hline
		\textbf{Como:} & Cliente \\ \hline
		\textbf{Me gustaría:} & poder aplicar el test de Shaffer para comparaciones múltiples \\ \hline
		\textbf{Ya que:} & se trata de una prueba multitest con mucha potencia y puede que rechace muchas más hipótesis \\ \hline
		\multirow{3}{11cm}{\textbf{C. Aceptación:}} & - El módulo de Python tiene implementado el test de Shaffer de comparaciones múltiples \\
		& - La plataforma muestra una sección para tests no paramétricos donde se puede aplicar la prueba \\
		& - El test devuelve los resultados propios del test, como los $\frac{K(K-1)}{2}$ $p-valores$ ajustados, estadísticos, resultados, etc. \\ \hline
		\textbf{\textbf{Prioridad:}} & alta \\ \hline
	\end{tabular}
\end{table}

% *****************HISTORIA USUARIO 20***************** %

\clearpage
Desglosando el Epic \textbf{EP-4:} Permitir realizar tests estadísticos no paramétricos para evaluar las condiciones paramétricas, se obtienen las siguientes historias de usuario:

\begin{table}[H]
	\begin{tabular}{| p{3cm}| p{11cm} |}
		\hline
		\multicolumn{2}{|c|}{\textbf{HU-20} - Realizar tests de normalidad} \\ \hline
		\textbf{Como:} & Cliente \\ \hline
		\textbf{Me gustaría:} & poder realizar distintos tests para ver si mis muestras de datos obtenidas por los algoritmos están distribuidas de forma normal \\ \hline
		\textbf{Ya que:} & esto ayuda a determinar si sobre mis datos puedo aplicar tests paramétricos o si por el contrario lo más adecuado (si no existe normalidad) es aplicar un test no paramétrico \\ \hline
		\multirow{2}{11cm}{\textbf{C. Aceptación:}} & - La plataforma muestra una sección para tests de condiciones paramétricas donde se pueden aplicar la pruebas de normalidad de Shapiro–Wilk, D’Agostino–Pearson y Kolmogorov–Smirnov para determinar si los datos siguen una distribución normal, siendo el test de Shapiro–Wilk el más potente y Kolmogorov–Smirnov el menos potente \\
		& - El test devuelve, para cada muestra de resultados obtenida por cada algoritmo, los estadísticos, $p-valores$ y resultados \\ \hline
		\textbf{\textbf{Prioridad:}} & alta \\ \hline
	\end{tabular}
\end{table}

% *****************HISTORIA USUARIO 21***************** %

\begin{table}[H]
	\begin{tabular}{| p{3cm}| p{11cm} |}
		\hline
		\multicolumn{2}{|c|}{\textbf{HU-21} - Realizar test de Levene} \\ \hline
		\textbf{Como:} & Cliente \\ \hline
		\textbf{Me gustaría:} & poder realizar el test de Levene para ver si mis muestras de datos obtenidas por
		los algoritmos están distribuidas de forma normal  \\ \hline
		\textbf{Ya que:} & se trata de una prueba para determinar la homocedasticidad, condición requerida para aplicar correctamente un test paramétrico o no paramétrico en caso de que los datos no presenten homocedasticidad \\ \hline
		\multirow{2}{11cm}{\textbf{C. Aceptación:}} & - La plataforma muestra una sección para tests de condiciones paramétricas donde se puede aplicar la prueba de homocedasticidad de Levene \\
		& - El test devuelve, el estadístico, el $p-valor$ y el resultado de la prueba \\ \hline
		\textbf{\textbf{Prioridad:}} & alta \\ \hline
	\end{tabular}
\end{table}

% *****************HISTORIA USUARIO 22***************** %

\newpage
Desglosando el Epic \textbf{EP-5:} Permitir gestionar un fichero, se obtienen las siguientes historias de usuario:

\begin{table}[H]
	\begin{tabular}{| p{3cm}| p{11cm} |}
		\hline
		\multicolumn{2}{|c|}{\textbf{HU-22} - Subir fichero de datos} \\ \hline
		\textbf{Como:} & Cliente \\ \hline
		\textbf{Me gustaría:} & poder subir en un fichero los resultados obtenidos por los algoritmos de aprendizaje \\ \hline
		\textbf{Ya que:} & es la forma más cómoda de proporcionar los datos y realizar los tests sobre ellos \\ \hline
		\multirow{2}{11cm}{\textbf{C. Aceptación:}} & - La plataforma muestra en la página principal un botón para subir el fichero \\
		& - En la parte superior de la plataforma también hay disponible un botón para poder realizar la subida de datos \\ \hline
		\textbf{\textbf{Prioridad:}} & alta \\ \hline
	\end{tabular}
\end{table}

% *****************HISTORIA USUARIO 23***************** %

\begin{table}[H]
	\begin{tabular}{| p{3cm}| p{11cm} |}
		\hline
		\multicolumn{2}{|c|}{\textbf{HU-23} - Consultar fichero de datos} \\ \hline
		\textbf{Como:} & Cliente \\ \hline
		\textbf{Me gustaría:} & poder consultar los datos de mi fichero de resultados obtenidos por los algoritmos \\ \hline
		\textbf{Ya que:} & es una forma de poder visualizar los datos sin tener que abrir el fichero en el escritorio local \\ \hline
		\multirow{2}{11cm}{\textbf{C. Aceptación:}} & - La plataforma en la parte superior muestra un botón para consultar el fichero en caso de que se haya subido uno previamente \\
		& - Al subir un fichero, la plataforma redirige automáticamente a la página de visualización del archivo \\ \hline
		\textbf{\textbf{Prioridad:}} & media \\ \hline
	\end{tabular}
\end{table}

% *****************HISTORIA USUARIO 24***************** %

Desglosando el Epic \textbf{EP-6:} Visualizar resultados tests, se obtienen las siguientes historias de usuario:

\begin{table}[H]
	\begin{tabular}{| p{3cm}| p{11cm} |}
		\hline
		\multicolumn{2}{|c|}{\textbf{HU-24} - Visualizar resultados de los test} \\ \hline
		\textbf{Como:} & Cliente \\ \hline
		\textbf{Me gustaría:} & poder ver los resultados en forma de tabla señalando el lugar concreto de la tabla por donde se pasa el ratón \\ \hline
		\textbf{Ya que:} & de esta forma, la lectura de resultados será más simple \\ \hline
		\multirow{2}{11cm}{\textbf{C. Aceptación:}} & - Los resultados se muestran en forma de tabla, teniendo en la primera fila de las columnas el nombre que identifica a los datos de la columna \\
		& - Cuando se pasa el ratón por un elemento de la tabla, toda la fila se destaca para poder visualizar mejor los datos relacionados \\ \hline
		\textbf{\textbf{Prioridad:}} & alta \\ \hline
	\end{tabular}
\end{table}

% *****************HISTORIA USUARIO 25***************** %

\begin{table}[H]
	\begin{tabular}{| p{3cm}| p{11cm} |}
		\hline
		\multicolumn{2}{|c|}{\textbf{HU-25} - Exportar resultados csv} \\ \hline
		\textbf{Como:} & Cliente \\ \hline
		\textbf{Me gustaría:} & poder exportar los resultados a fichero en formato csv \\ \hline
		\textbf{Ya que:} & así puedo guardar de forma automática los resultados en local en un formato sencillo y legible \\ \hline
		\multirow{2}{11cm}{\textbf{C. Aceptación:}} & - Cuando se muestran los resultados en una tabla encima de ésta aparece un botón para exportar a formato csv \\
		& - En caso de que se aplique un test de comparación POST-HOC también se incluirá un botón para exportar a csv los resultados de éste \\ \hline
		\textbf{\textbf{Prioridad:}} & media \\ \hline
	\end{tabular}
\end{table}

% *****************HISTORIA USUARIO 26***************** %

\begin{table}[H]
	\begin{tabular}{| p{3cm}| p{11cm} |}
		\hline
		\multicolumn{2}{|c|}{\textbf{HU-26} - Exportar resultados \LaTeX} \\ \hline
		\textbf{Como:} & Cliente \\ \hline
		\textbf{Me gustaría:} & poder exportar los resultados a fichero en formato \LaTeX \\ \hline
		\textbf{Ya que:} & así puedo guardar de forma automática los resultados para luego poder utilizarlos en un documento de \LaTeX sin necesidad de crear manualmente un tabla \\ \hline
		\multirow{2}{11cm}{\textbf{C. Aceptación:}} & - Cuando se muestran los resultados en una tabla encima de ésta aparece un botón para exportar a formato \LaTeX \\
		& - En caso de que se aplique un test de comparación POST-HOC también se incluirá un botón para exportar a \LaTeX los resultados de éste \\ \hline
		\textbf{\textbf{Prioridad:}} & media \\ \hline
	\end{tabular}
\end{table}

% *****************HISTORIA USUARIO 27***************** %

Desglosando el Epic \textbf{EP-7:} Permitir modificar opciones de los tests, se obtienen las siguientes historias de usuario:

\begin{table}[H]
	\begin{tabular}{| p{3cm}| p{11cm} |}
		\hline
		\multicolumn{2}{|c|}{\textbf{HU-27} - Seleccionar nivel de significancia} \\ \hline
		\textbf{Como:} & Cliente \\ \hline
		\textbf{Me gustaría:} & poder elegir el nivel de significancia de los tests \\ \hline
		\textbf{Ya que:} & el nivel es un parámetro muy importante a la hora de calcular los resultados de los tests y conviene que sea un parámetro modificable \\ \hline
		\textbf{C. Aceptación:} & - En la elección del test se da la posibilidad de seleccionar un nivel de significación. Por defecto se establece el más común: 0.05  \\ \hline
		\textbf{\textbf{Prioridad:}} & Alta \\ \hline
	\end{tabular}
\end{table}

% *****************HISTORIA USUARIO 28***************** %

\begin{table}[H]
	\begin{tabular}{| p{3cm}| p{11cm} |}
		\hline
		\multicolumn{2}{|c|}{\textbf{HU-28} - Seleccionar función objetivo} \\ \hline
		\textbf{Como:} & Cliente \\ \hline
		\textbf{Me gustaría:} & poder elegir la función objetivo de los algoritmos con los que voy a aplicar los tests \\ \hline
		\textbf{Ya que:} & esto modifica el orden del ranking en los tests no paramétricos de ranking \\ \hline
		\textbf{C. Aceptación:} & - En la elección del test se da la posibilidad de seleccionar minimización (establecido por defecto) o maximización  \\ \hline
		\textbf{\textbf{Prioridad:}} & media \\ \hline
	\end{tabular}
\end{table}

% *****************HISTORIA USUARIO 29***************** %

\clearpage
Desglosando el Epic \textbf{EP-8:} Proporcionar una interfaz usable, se obtienen las siguientes historias de usuario:

\begin{table}[H]
	\begin{tabular}{| p{3cm}| p{11cm} |}
		\hline
		\multicolumn{2}{|c|}{\textbf{HU-29} - Ver ayuda} \\ \hline
		\textbf{Como:} & Cliente \\ \hline
		\textbf{Me gustaría:} & poder acceder a la ayuda de la plataforma \\ \hline
		\textbf{Ya que:} & siempre es útil tener un acceso a información que puede ayudar a entender mejor los tests, y los conceptos relacionados con los resultados obtenidos, etc. \\ \hline
		\multirow{2}{11cm}{\textbf{C. Aceptación:}} & - En la parte superior derecha de la plataforma aparece un botón para acceder a la ayuda \\
		& - En los lugares donde sea preciso (en el panel de subida de fichero, en los tests, etc.) hay disponible un enlace al concepto relacionado en la ayuda \\ \hline
		\textbf{\textbf{Prioridad:}} & media \\ \hline
	\end{tabular}
\end{table}

% *****************HISTORIA USUARIO 30***************** %

\begin{table}[H]
	\begin{tabular}{| p{3cm}| p{11cm} |}
		\hline
		\multicolumn{2}{|c|}{\textbf{HU-30} - Recordar test de ranking} \\ \hline
		\textbf{Como:} & Cliente \\ \hline
		\textbf{Me gustaría:} & poder visualizar, cuando estoy eligiendo un test de comparación POST-HOC, el test de ranking previamente seleccionado \\ \hline
		\textbf{Ya que:} & así me podría dar cuenta si me equivoco en la elección \\ \hline
		\textbf{C. Aceptación:} & - En la selección del test POST-HOC aparece encima el test de ranking previamente seleccionado \\ \hline
		\textbf{\textbf{Prioridad:}} & baja \\ \hline
	\end{tabular}
\end{table}

% *****************HISTORIA USUARIO 31***************** %

\begin{table}[H]
	\begin{tabular}{| p{3cm}| p{11cm} |}
		\hline
		\multicolumn{2}{|c|}{\textbf{HU-31} - Avisar condiciones paramétricas} \\ \hline
		\textbf{Como:} & Cliente \\ \hline
		\textbf{Me gustaría:} & que la plataforma me diese un aviso en caso de que intente aplicar un test paramétrico sin haber hecho previamente las condiciones paramétricas \\ \hline
		\textbf{Ya que:} & aplicar un test paramétrico sobre datos que no siguen las suposiciones paramétricas no da resultados fiables \\ \hline
		\multirow{2}{11cm}{\textbf{C. Aceptación:}} & - la plataforma da un aviso si se intenta hacer un test paramétrico sin haber comprobado antes las condiciones sobre esos datos \\
		& - No se impide hacer el test si así lo desea el usuario \\ \hline
		\textbf{\textbf{Prioridad:}} & media \\ \hline
	\end{tabular}
\end{table}

% *****************HISTORIA USUARIO 32***************** %

\begin{table}[H]
	\begin{tabular}{| p{3cm}| p{11cm} |}
		\hline
		\multicolumn{2}{|c|}{\textbf{HU-32} - Ver breve información tests} \\ \hline
		\textbf{Como:} & Cliente \\ \hline
		\textbf{Me gustaría:} & debajo de las opciones de realizar los tests, apareciese una breve descripción de los mismos \\ \hline
		\textbf{Ya que:} & conviene tener, para el que no lo sepa, una mínima idea de qué hipótesis se contrastan en él \\ \hline
		\textbf{C. Aceptación:} & - Debajo de las opciones de los tests aparece una breve descripción de las hipótesis que se contrastan \\ \hline
		\textbf{\textbf{Prioridad:}} & media \\ \hline
	\end{tabular}
\end{table}

% *****************HISTORIA USUARIO 33***************** %

\begin{table}[H]
	\begin{tabular}{| p{3cm}| p{11cm} |}
		\hline
		\multicolumn{2}{|c|}{\textbf{HU-33} - Enlazar ficheros de ejemplo} \\ \hline
		\textbf{Como:} & Cliente \\ \hline
		\textbf{Me gustaría:} & que en la sección de ayuda de fichero hubiese enlaces a archivos de ejemplo \\ \hline
		\textbf{Ya que:} & así podría probar la aplicación sin necesidad de crear un archivo y además podría ver mejor cómo tiene que ser el formato del archivo de resultados a contrastar \\ \hline
		\textbf{C. Aceptación:} & - En la sección de ayuda de fichero aparecen enlaces a ficheros de ejemplo \\ \hline
		\textbf{\textbf{Prioridad:}} & baja \\ \hline
	\end{tabular}
\end{table}

% *****************HISTORIA USUARIO 34***************** %

\begin{table}[H]
	\begin{tabular}{| p{3cm}| p{11cm} |}
		\hline
		\multicolumn{2}{|c|}{\textbf{HU-34} - Aplicación en diferentes pestañas del navegador} \\ \hline
		\textbf{Como:} & Cliente \\ \hline
		\textbf{Me gustaría:} & poder utilizar de forma independiente la plataforma en diferentes pestañas del navegador \\ \hline
		\textbf{Ya que:} & así puedo estar trabajando a la vez con diferentes archivos \\ \hline
		\textbf{C. Aceptación:} & - La plataforma puede operar independientemente en diferentes pestañas con diferentes archivos de datos \\ \hline
		\textbf{\textbf{Prioridad:}} & alta \\ \hline
	\end{tabular}
\end{table}

% *****************HISTORIA USUARIO 35***************** %

\begin{table}[H]
	\begin{tabular}{| p{3cm}| p{11cm} |}
		\hline
		\multicolumn{2}{|c|}{\textbf{HU-35} - Diseño adaptable} \\ \hline
		\textbf{Como:} & Cliente \\ \hline
		\textbf{Me gustaría:} & poder utilizar la plataforma en dispositivos pequeños \\ \hline
		\textbf{Ya que:} & esto haría la plataforma más usable y podría trabajar en más dispositivos, como una tableta o móvil \\ \hline
		\textbf{C. Aceptación:} & - El diseño de la plataforma se adapta a dispositivos cuya resolución es más baja de los normal \\ \hline
		\textbf{\textbf{Prioridad:}} & media \\ \hline
	\end{tabular}
\end{table}

% *****************HISTORIA USUARIO 36***************** %

\begin{table}[H]
	\begin{tabular}{| p{3cm}| p{11cm} |}
		\hline
		\multicolumn{2}{|c|}{\textbf{HU-36} - Idioma} \\ \hline
		\textbf{Como:} & Cliente \\ \hline
		\textbf{Me gustaría:} & que el idioma de la aplicación fuese el inglés \\ \hline
		\textbf{Ya que:} & esto haría que la aplicación fuese mas fácilmente divulgable y potencialmente utilizada por más personas \\ \hline
		\textbf{C. Aceptación:} & - Los textos de la plataforma están escritos en inglés \\ \hline
		\textbf{\textbf{Prioridad:}} & baja \\ \hline
	\end{tabular}
\end{table}

% *****************HISTORIA USUARIO 37***************** %

\begin{table}[H]
	\begin{tabular}{| p{3cm}| p{11cm} |}
		\hline
		\multicolumn{2}{|c|}{\textbf{HU-37} - Flujo de trabajo} \\ \hline
		\textbf{Como:} & Cliente \\ \hline
		\textbf{Me gustaría:} & poder visualizar el flujo de trabajo de la aplicación de los tests al entrar en la plataforma \\ \hline
		\textbf{Ya que:} & esto me permitiría tener a primera vista una referencia clara de los pasos que debo seguir y las cosas que puedo hacer \\ \hline
		\multirow{2}{11cm}{\textbf{C. Aceptación:}} & - La página principal de la plataforma muestra el flujo de trabajo a través una imagen donde se detalla de forma esquemática el flujo \\
		& - En la página principal se muestra un botón para subir un fichero, primer paso del flujo de trabajo \\ \hline
		\textbf{\textbf{Prioridad:}} & baja \\ \hline
	\end{tabular}
\end{table}

% ----------------------------------------------------- %

%***************************************************************************************************************************